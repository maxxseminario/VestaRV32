% Generates an image of a QFN-style chip with a variable dimension and number 
% of pins.  It assumes the chip/cavity is square, which is often the case for
% even non-rectangular packages.

\documentclass{standalone}

\usepackage{tikz}

\begin{document}

% Use sans-serif font
\sffamily

% Change the x/y units to set the overall image size.  This will likely require
% updating the fontScale as well.
\begin{tikzpicture}[x=20mm, y=20mm]

    % User-edited dimensions.  All dimensions are in mm.
    \def\pinCount{48}       % The number of package pins
    \def\packageSize{7}     % For QFN, the outer dimension of the package
    \def\cavitySize{5.2}    % The size of the inner package cavity/maximum die size
    \def\fontScale{0.8}     % Controls size of pin number fonts--adjust if too large/small

    % Tweak these at your own risk.  Most control proportions and visual appearance.
    \def\pinCavityOffset{0.1}                       % Distance between the bond fingers and the pad cavity
    \def\wallThickness{0.2}                         % Thickness of the side walls of the plastic package
    \pgfmathsetmacro\pinWidth{0.6*360/\pinCount}    % Width of bond fingers in degrees--adjust to prevent finger overlap
    \pgfmathsetmacro\chamferSize{4*\wallThickness}  % Size of rounded corner regions
    \def\cornerReservedDegrees{5}                   % Angle of region in corners where no bond fingers are allowed
    \def\cornerGap{\pinCavityOffset}                % The separation gap of the bond finger blocks in each corner

    % Locate the southwest and northeast corners of the whole package.
    \pgfmathsetmacro\packageSW{-\packageSize/2}
    \pgfmathsetmacro\packageNE{\packageSize/2}

    % Locate the southwest and northeast corners of the central chip cavity.
    \pgfmathsetmacro\cavitySW{-\cavitySize/2}
    \pgfmathsetmacro\cavityNE{\cavitySize/2}

    % Other dimensions
    \pgfmathsetmacro\pinSW{\cavitySW-\pinCavityOffset}
    \pgfmathsetmacro\pinNE{\cavityNE+\pinCavityOffset}
    \pgfmathsetmacro\pinCountPerSide{\pinCount/4}

    % Draw the QFN Plastic Package Outline
    \draw (\packageSW,\packageSW) rectangle (\packageNE,\packageNE);

    % Draw the die in the center based on polygons exported from using klayout.
	\input{ezbond_die.tex}

    % Bondpad locations extracted from klayout.
    \input{ezbond_bondwire.tex}

    \begin{scope}

        % Use the same border drawn below to clip the bounds of the interior
        % shapes.  The actual shape is drawn afterward to prevent its width
        % from getting covered by the interior drawings.  Everything in this
        % scope will be clipped by this shape.
        \pgfmathsetmacro\wallThicknessInset{\wallThickness*0.5} % It is a thinner portion than the main wall
        \pgfmathsetmacro\wallThicknessB{\wallThickness+\wallThicknessInset}
        \pgfmathsetmacro\chamferSizeB{\chamferSize-\wallThicknessInset}
        \clip
           (\packageSW+\wallThicknessB,\packageNE-\wallThickness-\chamferSize) 
        -- (\packageSW+\wallThicknessB,\packageSW+\wallThickness+\chamferSize)
        arc[radius=\chamferSizeB, start angle=180, end angle=270]
        -- (\packageNE-\wallThickness-\chamferSize,\packageSW+\wallThicknessB)
        arc[radius=\chamferSizeB, start angle=-90, end angle=0]
        -- (\packageNE-\wallThicknessB,\packageNE-\wallThickness-\chamferSize)
        arc[radius=\chamferSizeB, start angle=0, end angle=90]
        -- (\packageSW+\wallThickness+\chamferSize,\packageNE-\wallThicknessB)
        -- cycle;

        % Draw the cavity bond fingers radially out from the chip center.  They are
        % slightly inset from the cavity edge, so a new calculated dimension is
        % used to locate their vertices.  For each bond finger, also add a pin
        % number and draw a bond wire.
        \foreach \side in {1,2,3,4} {

            % Draw the cavity boundary region on each size
            \draw[rotate={90*(\side-1)}]   (\packageSW,{\packageNE-\cornerGap}) -- (\packageSW,{\packageSW+\cornerGap}) -- (\cavitySW,{\cavitySW+\cornerGap}) -- (\cavitySW,{\cavityNE-\cornerGap}) -- cycle;

            \foreach \pinPos in {1,2,...,\pinCountPerSide} {

                % Angle of centerline of radial line out to pin location.  Start on 
                % the west edge, northernmost pin (just past 135 degrees).
                \pgfmathsetmacro\pinOffset{(135+\cornerReservedDegrees/2)}
                \pgfmathsetmacro\pinSpacing{(360-4*\cornerReservedDegrees)/\pinCount}
                \pgfmathsetmacro\pinAngle{\pinOffset+\pinSpacing*((\pinPos-1)+0.5)}

                % Find the angular wedge associated with this pin.  This will 
                % determine the width of the metal finger.
                \pgfmathsetmacro\pinAngleMin{\pinAngle-\pinWidth/2}
                \pgfmathsetmacro\pinAngleMax{\pinAngle+\pinWidth/2}

                % Find the points where this radial line intersects the package 
                % boundary and cavity boundary.  This will be used to draw the 
                % trapezoidal bonding area for the package pin.  Here we know
                % the angle and the either the x or y offset (depending on which
                % side of the chip the pin is on).
                \pgfmathsetmacro\pinA{(\packageSW)*tan(\pinAngleMin)}
                \pgfmathsetmacro\pinB{(\packageSW)*tan(\pinAngleMax)}
                \pgfmathsetmacro\pinC{(\pinSW)*tan(\pinAngleMax)}
                \pgfmathsetmacro\pinD{(\pinSW)*tan(\pinAngleMin)}

                % Draw the metal package pins, using rotate commands to prevent 
                % the need to recalculate the vertices.
                \draw[color=black, fill=white, rotate={90*(\side-1)}]   (\packageSW,\pinA) -- (\packageSW,\pinB) -- (\pinSW,\pinC) -- (\pinSW,\pinD) -- cycle;

                % Add pin labels. Pin number for other edges need to be calculated.
                \pgfmathsetmacro\pinLabelX{\packageSW*0.3+\pinSW*0.7} % Weighted average to set spacing near middle
                \pgfmathsetmacro\pinLabelY{(\pinLabelX)*tan(\pinAngle)}
                \pgfmathsetmacro\pinNum{int(\pinPos+(\side-1)*\pinCountPerSide)}
                \draw[rotate={90*(\side-1)}]   (\pinLabelX,\pinLabelY) node[scale=\fontScale] {\textbf{\pinNum}};

                % Draw bond wire.  Bondpad coordinates are centered at 0,0 and
                % absolute (all quadrants).  Pin finger coordinates are
                % calculated only for one edge (west) and then rotated.  This
                % complicates drawing a line between the two on all sides.
                % Rotate only the bond finger coordinates to convert them to
                % the final position in the drawing.  The bondpad coordinates
                % are already correct.
                \pgfmathsetmacro\pinWireX{\packageSW*0.1+\pinSW*0.9} % Weighted to fall near inner edge
                \pgfmathsetmacro\pinWireY{(\pinWireX)*tan(\pinAngle)}
                \pgfmathsetmacro\x{\bpx[\pinNum-1]}
                \pgfmathsetmacro\y{\bpy[\pinNum-1]}
                \coordinate (pinInitial) at (\pinWireX,\pinWireY);
                \coordinate (pinFinal) at ([rotate around={(90*(\side-1)):(0,0)}]pinInitial);
                \draw[color=blue, line width=1] (\x,\y) -- (pinFinal);
            }
        }

    \end{scope}

    % Draw the inner edge of the plastic package.  Start at pin 1 in the NW 
    % corner and go counter-clockwise.
    \draw
       (\packageSW+\wallThickness,\packageNE-\wallThickness-\chamferSize) 
    -- (\packageSW+\wallThickness,\packageSW+\wallThickness+\chamferSize)
    arc[radius=\chamferSize, start angle=180, end angle=270]
    -- (\packageNE-\wallThickness-\chamferSize,\packageSW+\wallThickness)
    arc[radius=\chamferSize, start angle=-90, end angle=0]
    -- (\packageNE-\wallThickness,\packageNE-\wallThickness-\chamferSize)
    arc[radius=\chamferSize, start angle=0, end angle=90]
    -- (\packageSW+\wallThickness+\chamferSize,\packageNE-\wallThickness)
    -- cycle;

    % Draw the second recessed inner edge of the plastic package.  Start at 
    % pin 1 in the NW corner and go counter-clockwise.
    \pgfmathsetmacro\wallThicknessInset{\wallThickness*0.5} % It is a thinner portion than the main wall
    \pgfmathsetmacro\wallThicknessB{\wallThickness+\wallThicknessInset}
    \pgfmathsetmacro\chamferSizeB{\chamferSize-\wallThicknessInset}
    \draw
       (\packageSW+\wallThicknessB,\packageNE-\wallThickness-\chamferSize) 
    -- (\packageSW+\wallThicknessB,\packageSW+\wallThickness+\chamferSize)
    arc[radius=\chamferSizeB, start angle=180, end angle=270]
    -- (\packageNE-\wallThickness-\chamferSize,\packageSW+\wallThicknessB)
    arc[radius=\chamferSizeB, start angle=-90, end angle=0]
    -- (\packageNE-\wallThicknessB,\packageNE-\wallThickness-\chamferSize)
    arc[radius=\chamferSizeB, start angle=0, end angle=90]
    -- (\packageSW+\wallThickness+\chamferSize,\packageNE-\wallThicknessB)
    -- cycle;

    % Draw arrow indicating the die orientation.  This script assumes 0 degree
    % rotation.
    \draw node[scale=0.5] (A) at (0,-0.1) {\textbf{GDS Orientation}};
    \draw[-latex,line width=1] (A) to (0,0.2);

\end{tikzpicture}

\end{document}
